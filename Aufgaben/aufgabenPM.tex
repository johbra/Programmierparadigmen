\documentclass[12pt]{scrartcl}
%\usepackage{times}
%\usepackage[T1]{fontenc}
\usepackage[utf8]{inputenc}
\usepackage{ngerman}
\usepackage[right]{eurosym} 
\usepackage{alltt}
\usepackage{graphicx}
\usepackage{scrpage2}
\usepackage{tocloft}
\usepackage{fancyvrb}
\usepackage{pdfsync}
\usepackage{textcomp}
\usepackage[extension=pdf]{hyperref}
\setlength{\unitlength}{5mm}

\def\figs{abbildungen/}

% Seitenstil

\renewcommand{\sectfont}{\rmfamily\bfseries}
\renewcommand{\headfont}{\normalfont}
\renewpagestyle{plain}%
    {(0pt,.5pt)%
    {}{}{FH NORDAKADEMIE \hfill Programmiermethodik \hfill Prof. Dr.-Ing. J. Brauer}(\textwidth,.5pt)}
    {(\textwidth,.5pt)%
    {}{}{Aufgaben \headmark\hfill Seite \pagemark}(\textwidth,0pt)}%\pagestyle{plain}
\pagestyle{plain}

%%%%%%%%%%%%%%%%%%%%%%%%%%%%%%%%%%%%%%%%%%%%%%%%%%
%%%%% Nutzung von tocloft für Aufgabenliste %%%%%%
%%%%% %%%%%%%%%%%%%%%%%%%%%%%%%%%%%%%%%%%%%%%%%%%%
\newcommand{\listaufgaben}{Aufgabenliste}

\newlistof{aufgabe}{agl}{\listaufgaben}

\newcommand{\aufgabe}[1]{%
\refstepcounter{aufgabe}
\bigskip
%für Übungsaufgaben gibt es keine Punkte:
%\par\noindent\textbf{\theaufgabe . Aufgabe  (#1 Punkte)}
\par\noindent\textbf{\theaufgabe . Aufgabe} #1
\addcontentsline{agl}{aufgabe}{\protect\numberline{\theaufgabe}#1}\par}

\cftpagenumbersoff{aufgabe}
\renewcommand{\cftafteragltitle}{%
\\[6pt]
\mbox{}{\normalfont\bfseries Nr. Punkte}}\renewcommand{\cftagltitlefont}{\large\bfseries}

\setlength{\cftafteragltitleskip}{0pt}
\newlength{\mylen}
\setlength{\mylen}{2em}
\renewcommand{\cftaufgabepresnum}{\hfill}%
\renewcommand{\cftaufgabeaftersnum}{\hspace*{\mylen}}%
\addtolength{\cftaufgabenumwidth}{\mylen}
%%%%%%%%%%%%%%%%%%%%%%%%%%%%%%%%%%%%%%%%%%%%%%%%%%%

\renewcommand{\labelenumi}{\alph{enumi})}
\renewcommand{\labelenumii}{\arabic{enumii})}

\begin{document}

\aufgabe{}

Schreiben Sie gemäß Entwurfsvorschrift I den Vertrag, die Zweckbestimmung
für eine Funktion auf, die den Rauminhalt eines Quaders berechnet,
für den Länge, Breite und Höhe gegeben ist.  Formulieren Sie drei
Beispielanwendungen als Tests, die das Verhalten der Funktion veranschaulichen.

\aufgabe{}Entwickeln Sie gemäß Entwurfsvorschrift I die folgenden Funktionen:

\begin{enumerate}
    \item Eine Funktion, die aus der Entfernung und der
    Geschwindigkeit zweier Züge die Zeit ermittelt, nach der die Züge
    sich treffen, wenn Sie sich auf einem gemeinsamen
    Streckenabschnitt von ihren jeweiligen Startpunkten aus sich
    aufeinander zu bewegen.


    \item Eine Funktion, die aus einem gegebenen Anfangskapital, einem
    Jahreszinssatz und einer in Monaten gemessenen Laufzeit das
    Endkapital ermittelt.  Die Zinsgutschrift erfolgt einmalig am Ende
    der Laufzeit.  Während der Laufzeit gibt es weder Einzahlungen
    noch Abhebungen.  

    \item Eine Funktion, die aus der Länge und Breite eines
    rechteckigen Fußbodens, die Anzahl der benötigten Fliesen
    berechnet, deren Größe ebenfalls durch Länge und Breite gegeben
    ist.
    
    Hinweis: Der Funktionsaufruf \texttt{(ceiling x)} liefert die
    kleinste ganze Zahl, die größer als x ist.

\end{enumerate}

\pagebreak
\aufgabe{Hilfsfunktionen und Variablendefinitionen} 

Schreiben Sie eine Funktion gemäß Entwurfsvorschrift I für die Berechnung des Volumens eines
Zylinders. Sie berechnet sich aus Grundfläche mal Höhe. Definieren Sie
eine Konstante für $\pi$, und Hilfsfunktionen für die Berechnung der
Grundfläche und des Quadrats einer Zahl. 

Testen Sie die Funktion z.\,B. mit
\begin{verbatim}
   (check-within (cylinder-volume 1 1) pi 0.01)
\end{verbatim}

Die Testprozedur 
\begin{verbatim}
   (check-within expr expr expr)
\end{verbatim}
funktioniert wie ähnlich wie \texttt{check-expect}, besitzt einen
dritten Ausdruck als Parameter, der als Wert eine Zahl $delta$
hat. Der Testfall überprüft, dass der Wert des ersten $expr$ maximal
um $delta$ von der entsprechenden Zahl im zweiten $expr$ abweicht.

\bigskip Beachten Sie die Angaben in \emph{moodle} bzgl. der Abgabe der Lösungen.  Sie dürfen 2er-Gruppen bilden.


\pagebreak
\aufgabe{Bedingte Funktionen}

Schreiben Sie ein Programm, das aus dem Bruttoeinkommen eines
Arbeitnehmers, das sich aus der Anzahl der Arbeitsstunden und seinem
Bruttostundenlohn ergibt, sein Nettoeinkommen durch Abzug der
Einkommenssteuer berechnet.  Die Einkommenssteuer wird dabei nach einem
steuererklärungaufbierdeckelgeeigneten Tarif ermittelt, der
folgendermaßen definiert ist:

\bigskip
\begin{tabular}{|c|c|}
    \hline
    \textbf{Einkommen} & \textbf{Steuersatz} [\%]  \\
    \hline
    $<= 5000$ & 0  \\
    \hline
    $> 5000\; und \leq 10000$ & 15  \\
    \hline
    $>10000\; und \leq 100000$ & 29  \\
    \hline
    $>100000$ & 64  \\
    \hline
\end{tabular}
\bigskip

Der Steuersatz gilt immer nur für die Einkommensanteile in dem
jeweiligen Intervall.

\medskip
Die Funktion \texttt{nettoeinkommen} soll nach folgendem Schema
aufrufbar sein:
\begin{verbatim}
      (nettoeinkommen anzahlArbeitsStunden stundenLohn)
\end{verbatim}
Hier noch ein paar Testvorgaben:
\begin{verbatim}
   (check-expect  (nettoeinkommen 1 5001)   (/ 500085 100))
   (check-expect  (nettoeinkommen 1 10001)  (/ 925071 100))
   (check-expect  (nettoeinkommen 1 100001) (/ 7315036 100))
\end{verbatim}
\noindent Hinweise: 

\begin{enumerate}
    \item Lesen Sie den Aufgabentext aufmerksam durch. Jeder Satz
      bedeutet etwas.

    \item Entwickeln Sie die Funktion gemäß Entwurfsvorschrift II.
    Benutzen Sie Hilfsfunktionen.  Senden Sie bis zum in der Vorlesung
    genannten Termin Ihr Racket-Programm an brauer@nordakademie.de.  Sie dürfen 2er-Gruppen
    bilden.
  \item Um das Rechnen mit inexakten Zahlen zu vermeiden, geben Sie
    Steuersätze nicht als Gleitkommazahlen wie 0.29 sondern als
    rationale Zahlen ein. In diesem Beispiel: (/ 29 100)

    \item Suchen Sie sich einen Abgeordneten, der Ihre Lösung in den
    Bundestag einbringt.  Wer diese Aufgabe bis zum 31.12.2013 gelöst
    hat, bekommt in der vierten Nachklausur einen Punkt gut
    geschrieben.
\end{enumerate}
 

\pagebreak
\aufgabe{Datenstrukturen}
\label{a5}
Gehen Sie für die Lösung der Aufgaben nach Entwurfsvorschrift III vor!

\begin{enumerate}
\item \begin{enumerate}
  \item Definieren Sie eine Datenstruktur für "`Zeitpunkte seit
    Mitternacht"', die aus den Komponenten \texttt{stunden},
    \texttt{minuten} und \texttt{sekunden} besteht.
  \item Entwickeln Sie eine Funktion \texttt{zeit->sekunden}, die eine
    Zeitpunkt-seit-Mittnacht-Struktur verarbeitet und die seit
    Mitternacht vergangenen Sekunden berechnet.
  \end{enumerate}

  \item\label{a5b} Definieren Sie eine Datenstruktur für Kreise, die durch
  \begin{itemize}
  \item die  Koordinaten des Mittelpunkts,
  \item den Radius und
  \item ihre Farbe
  \end{itemize}
  gegeben sind.

Für die Koordinaten ist die vordefinierte Datenstruktur \texttt{posn}
zu verwenden, die den folgenden Vertrag besitzt:
\begin{verbatim}
; Eine posn besteht aus
; - einer Zahl für die X-,
; - einer Zahl für die Y-Koordinate
(: make-posn (number number -> posn))
(: posn? (any -> boolean))
(: posn-x (posn -> number))
(: posn-y (posn -> number))
\end{verbatim}
Damit \texttt{posn} verwendet werden kann, muss in die erste Zeile
Ihres Programms \texttt{(require lang/posn)} eingetragen werden.

Farben sind durch Symbole\footnote{Symbole sind einfache Bezeichner
  mit vorangestelltem \textquotesingle.} wie \texttt{\textquotesingle red},
\texttt{\textquotesingle blue} etc. zu bezeichnen.
\begin{enumerate}
\item Schreiben Sie eine Funktion, die prüft, ob ein Punkt
  (\texttt{posn}) innerhalb eines Kreises liegt.
\item Kreise können auch gezeichnet werden. Die Prozeduren
  \texttt{draw-circle} und \texttt{draw-solid-disk}\footnote{Diese
    Prozduren stehen zur Verfügung, wenn Sie dem Kopf Ihres Programms
    die Zeile \texttt{(require htdp/draw)} hinzufügen.} zeichnen einen
  Kreis als Umriss bzw. ausgefüllt. Die Prozeduren verlangen jeweils
  drei Parameter: den Mittelpunkt, den Radius und die Farbe. Schreiben
  Sie eine Funktion, die einen Kreis als Argument und ein Symbol
  (\texttt{\textquotesingle umriss} oder \texttt{\textquotesingle
    gefuellt}) akzeptiert und ihn entsprechend malt.

Mit \texttt{(start 200 200)} wird eine Zeichenfläche mit 200 mal 200
Pixeln geöffnet.
  
\end{enumerate}

\end{enumerate}
\pagebreak
\aufgabe{Gemischte Daten}
Ein \emph{Mitarbeiter} ist entweder
\begin{itemize}
\item ein \emph{Festangestellter} oder
\item ein \emph{Werkstudent}
\end{itemize}

Ein \emph{Festangestellter} wird definiert durch
\begin{itemize}
\item seinen Namen,
\item sein Grundgehalt,
\item die im letzten Monat geleisteten Arbeitsstunden.
\end{itemize}

Ein \emph{Werkstudent} wird definiert durch
\begin{itemize}
\item seinen Namen,
\item seinen Stundenlohn,
\item die im letzten Monat geleisteten Arbeitsstunden.
\end{itemize}

Definieren Sie unter Berücksichtigung von Entwurfsvorschrift IV 
\begin{itemize}
\item geeignete Datenstrukturen für \emph{Mitarbeiter},
\item eine Funktionsschablone für Funktionen, die \emph{Mitarbeiter} verarbeiten. 
\end{itemize}

Entwickeln auf der Grundlage dieser Schablone eine Funktion, die den
Bruttomonatslohn eines Mitarbeiters berechnet. Bei \emph{Festangestellten}
berechnet sich der Monatslohn aus dem Grundgehalt zuzüglich
Überstundenentgelt. Überstunden sind die über die monatliche
Sollarbeitszeit (die als globale Konstante definiert wird)
hinausgehenden Arbeitsstunden. Der Stundenlohn pro Überstunde
berechnet sich aus dem Grundgehalt und der monatlichen Sollarbeitszeit
plus $25\%$. Minderstunden bleiben unberücksichtigt.


\pagebreak
\aufgabe{}\vspace{-0.0cm}
Werten Sie die folgenden Funktionsaufrufe aus:

%{\renewcommand{\arraystretch}{1.3}
\begin{tabular}%{|p{0.5cm}|p{5cm}|p{8cm}|}
    \hline
    a)	& (first '((A) B C D)) 	 &   \\
    \hline
    b)	& (rest '((A)(B C D))) 	 &   \\
    \hline
    c)	& (cons '(A B)  '(A B)) 	 &   \\
    \hline
    d)	& (cons 'A '()) 	 &   \\

    \hline
    e)	& (first '(((A))))	 &   \\
    \hline
    f)	& (rest '(((A)))) 	 &   \\
    \hline
    g)	& (cons '((A)) empty) 	 &   \\
    \hline
    h)	& (equal? 'X1 'X2) 	 &   \\
    \hline
    i)	& (equal=? '(X1) 'X2) 	 &   \\
    \hline
    j)	& (equal? '(X1) '(X2)) 	 &   \\
    \hline
    k)	& (pair? 'X1)	 &   \\
    \hline
    l)	& (pair? '(X1)) 	 &   \\
    \hline
    m)	& (pair? '())	 &   \\
    \hline
    n)	& (pair? '(empty))	 &   \\
    \hline
\end{tabular}}

% \pagebreak
% \aufgabe{X-Expressions}

% %\hyperdef{aufgabe}{formelnalssausdruecke}{Formel}
% In Teil 3 der Vorlesung wird eine Möglichkeit angegeben, einfache
%  arithmetische
%  \hyperref{../Vorlesung/PmFolien3}{aufgabe}{formelnalssausdruecke}{Formeln}
%  als
%  \hyperref{../Vorlesung/PmFolien3}{aufgabe}{formelnalssausdruecke}{S-Ausdrücke}
%  aufzuschreiben.
%  \begin{enumerate}
%  \item Entwerfen Sie eine Möglichkeit, derartige Formeln in XML aufzuschreiben.
%  \item Racket verfügt über einen Datentyp
%    \href{http://docs.racket-lang.org/xml/index.html#(tech._x._expression)}{\texttt{X-expression}}. Dabei
%    handelt es sich um S-Ausdrücke zur Repräsentation von
%    XML-Texten. Ob ein S-Ausdruck ein gültiger \texttt{ X-expression}
%    ist, kann mit der Funktion
% \href{http://docs.racket-lang.org/xml/index.html?q=xexpr%3F#(def._((lib._xml/main..rkt)._validate-xexpr))}{\texttt{validate-xexpr}}
%                                 überprüft werden.

%                                 \begin{quote}
%                                   \textbf{Hinweis}: Um die
%                                   XML-Funktionen von Racket nutzen zu
%                                   können, muss Ihre Racket-Datei mit
%                                   dem Ausdruck \texttt{(require xml)} beginnen.
%                                 \end{quote}
% Benutzen Sie die Funktion
% \href{http://docs.racket-lang.org/xml/index.html?q=xexpr%3F#(def._((lib._xml/main..rkt)._string-~3exexpr))}{\texttt{string->xexpr}},
%                                 um eine gemäß Ihrer XML-Darstellung aufgeschriebene
%                                 Formel in einen \texttt{X-expression}
%                                 zu verwandeln. Zum Beispiel liefert
%                                 der Aufruf von
% \begin{verbatim}
% (string->xexpr "<doc a='1' b='2'><bold>hi</bold> there!</doc>")
% \end{verbatim}
% den \texttt{X-expression}
% \begin{verbatim}
% (doc ((a "1") (b "2")) (bold "hi") " there!")
% \end{verbatim}
% \item Vergleichen Sie Ihre XML-Formeldarstellung mit dem Ergebnis der
%   Anwendung der Funktion
%   \href{http://docs.racket-lang.org/xml/index.html?q=xexpr-%3Estring&q=xexpr%3F#(def._((lib._xml/main..rkt)._xexpr-~3estring))}{\texttt{xexpr->string}} auf einen
%                                 Formel-S-Ausdruck gemäß  \hyperref{../Vorlesung/PmFolien3}{aufgabe}{formelnalssausdruecke}{Vorlesung}.


%  \end{enumerate}

% \pagebreak
\aufgabe{Semantik von Racket}

\begin{enumerate}
    \item  Werten Sie die folgenden Ausdrücke Schritt für Schritt aus:
    \vspace{-0.3cm}
    \begin{enumerate}
        \item  \texttt{(+ (* ( / 12 8) 2/3 ) (- 20 (sqrt 4)))}
    
        \item  
\begin{alltt}
    (cond
       [(= 0 0 ) false]
       [(> 0 1 ) (symbol=? 'a 'a)]
       [else ( = (/ 1 0 ) 9)])
\end{alltt}
    
        \item  
\begin{alltt}
    ( cond 
       [(= 2 0) false]
       [(> 2 1) (symbol=? 'a 'a )]
       [else (= ( / 1 2) 9)])
\end{alltt}
    \end{enumerate}

    \item  Gegeben sei die folgende Funktionsdefinition:
\vspace{-0.4cm}
\begin{alltt}
    ;; f:  number number -> number
    (define f
       (lambda (x y)
           (+ (* 3 x) (* y y))))
\end{alltt}
\vspace{-0.4cm}
Werten Sie die folgenden Ausdrücke Schritt für Schritt aus:
\vspace{-0.3cm}
\begin{enumerate}
    \item  \texttt{(+ (f 1 2) (f 2 1))}

    \item  \texttt{(f 1 (* 2 3))}

    \item  \texttt{(f (f 1 (* 2 3)) 19)}
\end{enumerate}
\end{enumerate}

\pagebreak
\aufgabe{Anwendungen der  Entwurfsvorschrift V} 

\begin{enumerate}
    \item Die Funktion \texttt{sum} liefere, angewendet auf eine Liste von
    numerischen Atomen \texttt{x}, die Summe der Elemente.

    \item Die Funktion \texttt{prod} liefere, angewendet auf eine Liste von
    numerischen Atomen \texttt{x}, das Produkt der Elemente.
 
    \item Die Funktion \texttt{maximum} liefere, angewendet auf eine Liste von
    numerischen Atomen \texttt{x}, das Maximum der Elemente.

    \item Die Funktion \texttt{enthaelt?} beantworte, angewendet auf
    ein Symbol und eine Liste von Symbolen, die Frage, ob das Symbol
    in der Liste enthalten ist oder nicht
 
\end{enumerate}

\aufgabe{Anwendungen der  Entwurfsvorschrift V} 

Schreiben Sie eine Funktion \texttt{(declist x)}, die aus einer Liste \texttt{x} von
Integers eine neue Liste berechnet, deren Elemente um 1 kleiner sind,
als die der ursprünglichen Liste:

{\ttfamily
\begin{center}
    \begin{tabular}{|l|l|}
        
        \hline
        x & (declist x)  \\
        \hline
        (2 5 7) & (1 4 6)  \\
        \hline
        empty & empty  \\
        \hline
    \end{tabular}
    
\end{center}
}

\bigskip Beachten Sie die Angaben in Moodle bzgl. der Abgabe der Lösungen.  Sie dürfen 2er-Gruppen bilden.
 
\pagebreak
\aufgabe{Listenverarbeitung}
\begin{enumerate}
    \item Definieren Sie eine Funktion \texttt{(flatten x)}, die als
    Argument eine Liste \texttt{x} mit beliebig tief geschachtelten
    Unterlisten hat und als Ergebnis eine Liste von Atomen liefern
    soll mit der Eigenschaft, dass alle Atome, die in x vorkommen auch
    in \texttt{(flatten x)} in derselben Reihenfolge vorkommen:
    {\ttfamily
    \begin{center}
        \begin{tabular}{|l|l|}
            
            \hline
            x & (flatten x)  \\
            \hline
            (A (B C) D) & (A B C D)  \\
            \hline
            (((A B) C)(D E)) & (A B C D E)   \\
            \hline
            ((((A)))) & (A) \\
            \hline
        \end{tabular}
        
    \end{center}
    }
    \paragraph{Hinweis:} Definieren Sie zuerst in der bekannten Art
    und Weise eine rekursive Datenstruktur für \glqq geschachtelte\grqq{} 
    Listen. Wenden Sie dann die passende \textbf{Entwurfsvorschrift} an.

    \item Schreiben Sie eine Funktion \texttt{(frequencies x)}, die
    aus einer Liste \texttt{x} von Atomen eine Liste von
    zwei-elementigen Strukturen (define-record-procedures) erzeugt:
    Dabei ist das erste Element das Atom aus \texttt{x}, das zweite
    Element die Häufigkeit des Auftretens in
    \texttt{x}.  Die Reihenfolge der Strukturen in der
    Ergebnisliste ist belanglos.
    {\ttfamily
    \begin{center}
        \begin{tabular}{|l|l|}
            
            \hline
            x & (frequencies x)  \\
            \hline
            (A B A B A C A) & (<A  4> <B  2> <C  1>)  \\
            \hline
            empty & empty  \\
            \hline
        \end{tabular}
        
    \end{center}
    }
\verb+<A 4>+ stehe für einen Record mit den Elementen \texttt{A} und $4$
\end{enumerate}

\aufgabe{Mengenoperationen}
Entwerfen Sie einen Satz von Funktionen für die Mengenoperationen
VEREINIGUNG, DURCHSCHNITT, DIFFERENZ, wobei Mengen als Listen von
Atomen ohne Wiederholungen repräsentiert werden sollen.
\pagebreak

\aufgabe{Verarbeitung von zwei Listen}

Die in den folgenden Aufgaben zu entwickelnden Funktionen
haben alle 2 Listen-Parameter.  Lösen Sie diese Aufgaben unter
Anwendung von Entwurfsvorschrift V. Überlegen Sie dabei, ob für die
Erstellung der Funktionsschablone der Zugriff auf das erste Element
und die Restliste hinsichtlich des ersten, des zweiten oder beider
Parameter vorgenommen werden muss.
\begin{enumerate}
    \item  Schreiben Sie ein Funktion \texttt{concatenate}, die zwei
    Listen von Symbolen aneinander hängt. Beispiel: \hfill \\
   \texttt{ (concatenate '(a b c) '(d e f)) => '(a b c d e f)}

    \item Schreiben Sie eine Funktion \texttt{mult-2-num-lists}, die
    zwei gleich lange Listen mit Zahlen zu einer Liste verarbeitet, die 
    die Produkte der korrespondierenden Elemente der Argumentlisten
    enthält. Beispiel: \hfill \\
    \texttt{(mult-2-num-lists '(2 3 4) '(7 8 9)) => '(14 24 36)}

    \item Entwickeln Sie eine Funktion \texttt{merge}, die 2 Listen von 
    Zahlen verarbeitet, die aufsteigend sortiert sind. Sie liefert
    eine sortierte Liste von Zahlen, die alle Zahlen aus den beiden
    Argumentlisten enthält. Wenn Zahlen in den Argumentliste mehrfach 
    vorkommen, sollen Sie auch in der Ergebnisliste entsprechend oft
    auftauchen.  Beispiel: \hfill \\
    \texttt{(merge '(2 5 7) '(1 3 5 9)) => '(1 2 3 5 5 7 9) }
\end{enumerate}

%%%%%%%%%%%%%%%%%%%%%%%%%%%%%

\aufgabe{Aufgaben zum Datentyp \texttt{Nat}}

    \begin{enumerate}
        \item Schreiben Sie eine Funktion \texttt{repeat}, die eine
        natürliche Zahl $n\in Nat$ (Definition des Datentyps $Nat$
        s. Vorlesung) und ein Symbol $s$ als Argumente
        nimmt und eine Liste mit dem $n$-maligen Auftreten von $s$
        erzeugt.
        
        \item  Schreiben Sie eine  Funktion \texttt{Int->Nat}, die
        eine beliebige positive ganze Zahl in eine äquivalentes Element 
        des Datentyps $Nat$ verwandelt, z.\,B.: \\
        \texttt{(Int->Nat 3) => (succ (succ (succ zero)))}
    
        \item  Schreiben Sie eine  Funktion \texttt{Nat->Int}, die
        eine natürliche Zahl $n\in Nat$ in eine \glqq normale\grqq{}
        Racket-Number verwandelt, z.\,B.: \\
        \texttt{(Nat->Int (succ (succ (succ zero)))) => 3}
    
        \item  Schreiben Sie eine  Funktion \\
        \texttt{;; times: Nat Nat -> Nat} \\
        , die zwei natürliche Zahlen multipliziert.
    \end{enumerate}
    
    
 \bigskip Beachten Sie die Angaben in Moodle bzgl. der Abgabe der Lösungen.  Sie dürfen 2er-Gruppen bilden.


\pagebreak
\aufgabe{Symbolische Differentiation}

Es sollen Ausdrücke abgeleitet werden, die nur aus Konstanten, Variablen
und den Operationen $+$ und $\cdot$ bestehen.

Sei $D_x$ die partielle Ableitung einer Funktion $f$ nach $x$, dann
gelten folgende Regeln:
\begin{itemize}
\item    $D_x(x) = 1$
\item    $D_x(y) = 0$, $y\neq x$, sei $y$ eine Konstante oder Variable
\item    $D_x(e_1 + e_2) = D_x(e_1)+D_x(e_2)$ (Summenregel)
\item    $D_x(e_1\cdot e_2) = e_1\cdot D_x(e_2) + e_2\cdot D_x(e_1)$ (Produktregel)
\end{itemize}
Repräsentation der Formeln:
\begin{itemize}
\item    Konstante: numerisches Atom
\item    Variable: symbolisches Atom
\item    $e_1 + e_2$: (ADD $e_1$ $e_2$)
\item    $e_1 \cdot e_2$: (MUL $e_1$ $e_2$)
\end{itemize}

Hinweise:
\begin{itemize}
    \item Definieren Sie zur Erzeugung von Formeln geeignete
    Hilfsfunktionen!

    \item Wenn eine Formel nicht korrekt aufgebaut ist, kann das
    Symbol \texttt{'ERROR} zurückgeliefert werden, das
    möglicherweise in einem korrekten Teil der Formel
    eingeschachtelt erscheint.

    \item Machen Sie ausgiebig von lokalen Definitionen Gebrauch.
\end{itemize}

\pagebreak
\aufgabe{Anwendung der Funktion \texttt{reduce}}
\label{Anwendungreduce}
Definieren Sie unter Verwendung von \texttt{reduce} (s. Vorlesung) 
eine Funktion \texttt{sort}, die, angewendet auf eine Liste von 
Zahlen, diese Liste absteigend sortiert.

\aufgabe{Verallgermeinerung von \texttt{sort}}
Modifizieren Sie die Funktion \texttt{sort} aus 
Aufgabe~\ref{Anwendungreduce} so, dass durch einen zusätzlichen 
Parameter die Sortierreihenfolge bestimmt werden kann.

\aufgabe{Anwendung von \texttt{mapp}}
Was liefern die folgenden Ausdrücke:
\begin{enumerate}
    \item  \verb|((mapp abs) '( 4 -7 3))|

    \item  \verb|(define betraege (mapp abs))|

    \item  \verb|(betraege '(4 -7 -3))|
    
    \item \verb|(reduce * ((mapp abs) '( 4 -7 3)) 1)|
\end{enumerate}

\aufgabe{Vertrag von \texttt{mapp}}
Wie lautet er?

\aufgabe{Anwendung von \texttt{mapp}}
Erklären Sie die Auswertung des Ausdrucks 
\verb|((mapp abs) '( 4 -7 3))| 
mithilfe des Ersetzungsmodells für Funktionsanwendungen. Vergleiche 
dazu Teil 5 der Vorlesungsfolien.

\pagebreak
\aufgabe{Hilfsfunktionen mit akkumulierenden Parametern}
\begin{enumerate}
    \item Schreiben Sie die Funktion \texttt{sum}, die die Summe der
    Elemente einer Liste von Zahlen berechnet, unter Benutzung einer
    Hilfsfunktion mit akkumulierendem Parameter.  Verwenden Sie die
    Funktiosnschablone aus der Vorlesung.  Formulieren Sie die
    Akkumulatorinvariante.

    \item  Gegeben ist ein Weg in einem ungerichteten Graphen, dessen 
    Knoten Orte repräsentieren und dessen Kanten mit den Entfernungen 
    zwischen den Orten attributiert sind, z.\,B. so:

    \includegraphics[scale=0.8]{\figs entfernungen.pdf}

    
    Entwickeln Sie eine Funktion, die aus einer Liste mit relativen 
    Entfernungen eine Liste mit den absoluten Entfernungen der Orte 
    vom Ursprungsort berechnet. Für den obigen Graphen soll also aus 
    der Liste (120 90 70 65) die Liste (120 210 280 345) werden.
    \begin{enumerate}
        \item  Entwickeln Sie zunächst eine Funktion (ggf. mit 
        Hilfsfunktion) nach der 
        bekannten Methode (ohne akkumulierende Parameter).
    
        \item Diskutieren Sie, warum eine Hilfsfunktion mit
        akkumulierendem Parameter sinnvoll ist.
    
        \item  Entwickeln Sie eine solche.
    \end{enumerate}
    
    \item Definieren Sie eine Funktion \verb|(singletons x)|, die als Argument
    eine Liste von den Atomen \texttt{x} hat und als Ergebnis eine Liste von den
    Atomen liefern soll, die in \texttt{x} genau einmal auftreten.

    \item Modifizieren die Funktion \verb|(singletons x)| so, dass zwei
    akkumulierende Parameter verwendet werden.  Der eine soll zum
    Akkumulieren der Atome, die genau einmal in \texttt{x} auftreten, dienen,
    der andere zum Akkumulieren der Atome, die mehrmals in \texttt{x}
    auftreten.


\end{enumerate}
\pagebreak
\aufgabe{Beweis einer rekursiven Funktion}

Gegeben sei folgende Racket-Funktion
\begin{alltt}
    (define f
      (lambda (n)
        (cond
          [(= n 0) 0]
          [else (+ (f (- n 1))
                   (/ 1 (* n (+ n 1))))])))
\end{alltt}
Zeigen Sie, dass der Aufruf \verb|(f n)| die Zahl
\[f(n)=\frac{n}{n+1}\]
berechnet.


\pagebreak
\aufgabe{Umgebungsmodell}

Grundlage:
\begin{alltt}
;; nimmt einen Betrag als Startkapital eines Kontos
;; und erzeugt eine "belaste-Funktion"
(: erzeuge-konto (number 
                -> (number -> (mixed number symbol))))
(define erzeuge-konto
  (lambda (konto)
    (lambda (betrag)
      (cond
        [(>= konto betrag)
         (begin
           (set! konto (- konto betrag))
           konto)]
        [else 'konto-ueberzogen]))))
\end{alltt}

\begin{enumerate}
    \item  Wie sieht das Umgebungsdiagramm aus, wenn zwei Konten 
    angelegt werden?
\begin{alltt}
(define konto1 (erzeuge-konto 100))
(define konto2 (erzeuge-konto 200))
\end{alltt}

    
   \item  Stellen Sie die Auswertung von (konto2 120) dar!
    \item  Wie sieht das Umgebungsdiagramm aus für?
 \begin{alltt}
 (define konto1 (erzeuge-konto 100))
 (define konto2 konto1)
 \end{alltt}

\end{enumerate}

\pagebreak
\aufgabe{Message passing style / Umgebungsmodell}
\begin{enumerate}
    \item  Betrachten Sie die folgende Variante der Prozedur 
    \texttt{erzeuge-konto}. Machen Sie sich ihre Wirkungsweise klar.
\begin{alltt}
(: erzeuge-konto 
   (number -> (symbol -> (number -> (mixed number symbol)))))
;; nimmt einen Betrag als Startkapital und erzeugt ein "Konto-Object"
(define erzeuge-konto
  (lambda (konto)
    (letrec
        ([;; belaste: (number -> (mixed number symbol))
          ;; Effekt: bucht vom konto betrag ab, liefert neuen 
          ;; Kontostand als Resultat, falls Konto nicht ueberzogen
          belaste
          (lambda (betrag)
            (cond
              [(>= konto betrag)
               (begin
                 (set! konto (- konto betrag))
                 konto)]
              [else 'konto-ueberzogen]))]
         [;; schreibegut: (number -> number)
          ;; Effekt: schreibt konto betrag gut
          ;; liefert neuen Kontostand als Resultat
          schreibegut
          (lambda (betrag)
            (begin 
              (set! konto (+ konto betrag))
              konto))]
         [;; verteile: (number -> (mixed number symbol))
          ;; verwaltet die von Konten verstandenen Nachrichten
          verteile
          (lambda (nachricht)
            (cond
              [(equal? nachricht 'belaste) belaste]
              [(equal? nachricht 'schreibegut) schreibegut]
              [else (violation "unbekannte Nachricht")]))])
      verteile)))
\end{alltt}

    \item  Wie sieht das Umgebungsdiagramm für die folgende 
    Ausdruckssequenz aus:
\begin{alltt}
(define konto (erzeuge-konto 200))
((konto 'schreibegut) 60)
((konto 'belaste) 120)
\end{alltt}

% s. SICP Ex. 3.11
\end{enumerate}

\pagebreak
\aufgabe{Umwandlung von ADTstack-Termen}
Welchen stack (gemäß Spezifikation aus der Vorlesung) repräsentiert der Term
\begin{alltt}
    push( pop(push(push(createstack, c), b)), a)
\end{alltt}
\begin{enumerate}
    \item  Geben Sie eine graphische Darstellung des resultierenden stacks an.

    \item Formen Sie den Term durch Anwendung der Gleichungen solange
    um, bis er nur noch die Operation \texttt{push} und \texttt{createstack}
    enthält.

\end{enumerate}

\aufgabe{Tupel-Spezifikation}
Spezifizieren Sie einen Datentyp ADTtupel.  Ein Tupel sei ein
Paar von items.  Definieren Sie eine Operation für das Erzeugen eines
Tupels aus zwei items, sowie je eine Operation für den Zugriff auf die
beiden Komponenten.

\aufgabe{Mengen-Spezifikation}
\label{mengenspezifikation}
Geben Sie die algebraische Spezifikation für einen abstrakten
Datentyp ADTset an, der eine Menge von Datenelementen (items) mit
folgenden Operationen beschreibt:

\begin{itemize}
    \item  \texttt{emptyset}: liefert die leere Menge

    \item  \texttt{makeset(i)}: liefert die Menge, deren einziges Element i ist

    \item  \texttt{union(s, s´)}: vereinigt die Mengen s und s´

    \item  \texttt{intersect(s, s´)}: bildet den Durchschnitt der Mengen s und s´

    \item  \texttt{isin(s, i)}: liefert TRUE, wenn i Element von s, sonst FALSE
\end{itemize}
Hinweis: In der Spezifikation von item sei eine Operation
\begin{alltt}
        eqitem: item, item --> bool
\end{alltt}
spezifiziert, die zwei items auf Gleichheit prüft.  Diese Operation
kann für die Spezifikation verwendet werden.

\aufgabe{Umwandlung von Set-Termen}
Formen Sie die folgenden Terme der Spezifikation von Aufgabe \ref{mengenspezifikation}
durch Anwendung der Gleichungen schrittweise um, bis keine
Gleichung mehr anwendbar ist.
\begin{enumerate}
    \item  \texttt{isin( union( union( emptyset, makeset(i)), makeset(j)), i)}

    \item  \verb|intersect(union(union(makeset(i1), makeset(i2)), makeset(i3)),|\\
           \verb|          union( makeset(i2), makeset(i4)))|
\end{enumerate}

Die Größen i, j, i1, i2, i3 und i4 seien paarweise verschiedene
Elemente von item.

\pagebreak
\aufgabe{Rationale Zahlen} 
Mit den folgenden Racket-Funktionen wird
eine \glqq abstrakte Implementierung\grqq{} für Brüche (rationale Zahlen)
vorgenommen:

\begin{alltt}
    (define (add-rat x y)
      (make-rat (+ (* (numer x) (denom y))
                   (* (numer y) (denom x)))
                (* (denom x) (denom y))))
    
    (define (sub-rat x y)
      (make-rat (- (* (numer x) (denom y))
                   (* (numer y) (denom x)))
                (* (denom x) (denom y))))
    
    (define (mul-rat x y)
      (make-rat (* (numer x) (numer y))
                (* (denom x) (denom y))))
    
    (define (div-rat x y)
      (make-rat (* (numer x) (denom y))
                (* (denom x) (numer y))))
    
    (define (equal-rat? x y)
      (= (* (numer x) (denom y))
         (* (numer y) (denom x))))
    
    (define (make-rat numer denom)
      (cons 'make-rat (cons numer (cons denom '()))))
    
    (define (numer rat) 
      (cadr rat))
    
    (define (denom rat) 
      (caddr rat))
    
    (define (print-rat x)
      (newline)
      (display (numer x))
      (display "/")
      (display (denom x)))
\end{alltt}

Zum Thema \glqq abstrakte Implementierungen\grqq{} vergleichen Sie 
auch die Aufgabe \emph{Aufgaben zum Datentyp \texttt{Nat}} (Nr. 10) 
bzw. Vorlesung Teil 3 (Natürliche Zahlen als rekursive Datenstruktur).

\begin{enumerate}
    \item  Machen Sie sich die Wirkungen dieser Funktionen klar.  Sie können Sie auch erproben.  Sie finden sie in der Rubrik Beispiele in moodle. \\

Hinweise:
\begin{itemize}
    \item Zum Testen wählen Sie in DrRacket die Spracheinstellung
    Plt-Text(MzRacket)-Kombo

    \item Die Funktion \texttt{cadr} liefert das zweite,
    \texttt{caddr} das dritte Element einer Liste.
\end{itemize}
    \item Ergänzen Sie das Racket-Programm so, dass Brüche durch die
    Funktion \texttt{make-rat} bei der Erzeugung gekürzt werden.
    
    \item Schreiben Sie eine algebraische Spezifikation für rationale 
    Zahlen. Das Vorhandensein einer Spezifikation für ganze Zahlen 
    darf dabei vorausgesetzt werden
\end{enumerate}

\aufgabe{Abstrakte Implementierung für Warteschlangen}

Schreiben Sie für die algebraische Spezifikation für Warteschlangen
aus dem Skript Funktionen als "`algebraische Implementierung"' nach dem Muster der Funktionen für den Datentyp \textbf{Nat} aus der Vorlesung bzw.  nach dem Muster der vorangegangenen Aufgabe.  Überlegen Sie sich zusätzlich, wie eine
Funktion \texttt{print-queue} aussehen könnte, die Queues lesbar darzustellen
erlaubt (vgl. die Funktion \texttt{print-rat}).

\aufgabe{Alternative Mengen-Spezifikation}
% Vergleiche hierzu Abschnitt 8.7 in Klären/Sperber
\begin{enumerate}
    \item  Geben Sie die algebraische Spezifikation für einen abstrakten
Datentyp ADTset an, der eine Menge von Datenelementen (items) mit
folgenden Operationen beschreibt:

\begin{itemize}
    \item  \texttt{emptyset}: liefert die leere Menge

    \item  \texttt{insert(i, s)}: fügt der Menge s das Element i hinzu

    \item  \texttt{member(i, s)}: liefert TRUE, wenn i Element von s, sonst FALSE
\end{itemize}
In der Spezifikation von item sei eine Operation
\begin{alltt}
        eqitem: item, item --> bool
\end{alltt}
spezifiziert, die zwei items auf Gleichheit prüft.  Diese Operation
kann für die Spezfikation verwendet werden.

    \item  Geben Sie zwei verschiedene Terme aus aufbauenden 
    Operationen an, die insofern die gleiche Menge darstellen, als 
    \texttt{member}, angewendet auf diese Mengen, für die gleichen 
    items true liefert.

    \item  Welche Gleichungen müssen Sie hinzufügen, um festzuhalten, 
    dass weder die Reihenfolge der \texttt{insert}-Aufrufe noch die 
    Existenz zu Dubletten eine Rolle spielt, um zwei Mengen als 
    gleich ansehen zu können.
    
    \item Erstellen Sie eine konkrete Implementierung für den 
    Datentyp in Racket unter Verwendung von Listen.
\end{enumerate}


\end{document}
% LocalWords:  figurenverarbeitende Festangestellter
