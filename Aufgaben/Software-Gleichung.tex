Die Software-Gleichung von Larry Putnam beschreibt den phänomenologischen Zusammenhang zwischen der Produktgröße $G$ (gemessen in \emph{lines of code}), dem Aufwand $A$ (gemessen Personenstunden), einem Technologiefaktor $P$ und der Projektdauer $t$: 
$$ G = P \cdot A^{\frac{1}{3}} \cdot  t^{\frac{4}{3}}$$ 

Entwickeln Sie eine Lösung dieser Gleichung mithilfe des aus Vorlesung bekannten Constraint-Programming-Systems. Die Existenz der Basis-Bausteine für die Addition, Multiplikation und Potenzbildung darf vorausgesetzt werden.

\begin{parts}
\part [5]

Implementieren Sie die Gleichung als Schaltbild aus den CPS-Grund\-elementen.

\part[3]

Setzen Sie diese Schaltbild in eine Scheme-Prozedur um.
\end{parts}
