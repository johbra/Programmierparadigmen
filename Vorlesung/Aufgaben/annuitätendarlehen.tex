\aufgabe{Annuitätendarlehen (Constraint Programming)}

Die folgende Formel beschreibt für ein Annuitätendarlehen den 
Zusammenhang zwischen der Höhe der monatliche Rückzahlungsrate $P$, 
der Darlehenssumme $A$, des monatlichen Zinssatzes  $R$ (= Jahreszinssatz/12)
und der Laufzeit in Monaten $N$.

$$P(1+R)^{N}-P=A\cdot R(1+R)^{N}$$
 
Schreiben Sie für die Formel eine Constraint-Programming-Spezifikation mit 
Hilfe des aus der Vorlesung bekannten Constraint-propagation-Systems. 
Die Existenz der primitiven Constraints \texttt{adder}, 
\texttt{multiplier} und \texttt{exponent} darf vorausgesetzt werden.

Gehen Sie wie folgt vor:
\begin{enumerate}

    \item  Zeichnen Sie ein \glqq Schaltbild\grqq{} für die 
    Formel!

    \item  Setzen Sie dann das Schaltbild in eine Scheme-Lösung nach 
    dem Vorbild des Celsius-Fahrenheit-Konverters um!

\end{enumerate}
