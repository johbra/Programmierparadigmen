In der Differenzkalkulation kann der Zusammenhang zwischen dem
Verkaufspreis ($vp$), dem Bezugspreis ($bp$) und dem
Kalkulationszuschlag in Prozent ($kp$) durch die folgende Gleichung
beschrieben werden: 
$$ \frac{vp}{bp} = 1 + \frac{kp}{100}$$ 

Entwickeln Sie eine Lösung dieser Gleichung mithilfe des aus Vorlesung
bekannten Con\-straint-Programming-Systems. Für die arithmetischen
Operationen sind nur die Basis-Bausteine für die Addition und die
Multiplikation zu verwenden.

\begin{parts}
\part [3]

Implementieren Sie die Gleichung als Schaltbild aus den CPS-Grund\-elementen!

\part[3]

Setzen Sie dieses Schaltbild in eine Scheme-Prozedur um!
\end{parts}